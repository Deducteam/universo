\documentclass{article}

\usepackage[utf8]{inputenc}

\usepackage{amsmath}

\usepackage{amsthm}

\usepackage{amssymb}

\usepackage{mathpartir}

\usepackage{mdframed}

\newtheorem{definition}{Definition}

\newtheorem{lemma}{Lemma}

\newcommand{\lfmt}{\ensuremath{\lambda\Pi\text{-modulo theory}}}

\newcommand{\defn}{\ensuremath{:=}}

\newcommand{\tabs}[3]{\ensuremath{\lambda{#1}\,{:}\,{#2}.\,{#3}}}

\newcommand{\tapp}[2]{\ensuremath{{#1}~{#2}}}

\newcommand{\tpi}[3]{\ensuremath{\Pi{#1}\,{:}\,{#2}.\,{#3}}}

\newcommand{\ttype}{\ensuremath{\mathbf{Type}}}

\newcommand{\tkind}{\ensuremath{\mathbf{Kind}}}

\begin{document}

\section{The algorithm}

Minimization of universes can be presented in an abstract way on a simpler theory than the one implemented in Coq. We are going to use a variant of the Calculus of Constructions with Universes as presented in~\ref{fig:sttsyntax}. Then, in TODO, we will present how we can extend this algorithm for a theory implemented in a system such as Coq or Matita.

\begin{figure}
  \centering
  \begin{tabular}{lccl}
    \textbf{Variables} & \(x,y,z,\dots\) & & \\
    \textbf{Terms} & \(A,B,t,u\) & \(\defn\) & \(x~|~\tpi{x}{A}{B} ~|~\tabs{x}{A}{u} ~|~ \tapp{t}{u} \)\\

  \end{tabular}
  \caption{\lfmt{}}
  \label{fig:sttsyntax}
\end{figure}
\end{document}